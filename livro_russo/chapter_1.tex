\documentclass{amsart}
\usepackage{amsmath}
\usepackage[brazil]{babel}
\author{Clovis W. Bertholini Sb.}
\title{Problems and Exercises of Mathematical Analysis}
\date{\today}
\begin{document}
\maketitle
\textbf{1.} Shows that
\begin{center}
$a,b \in \mathbb{R} \Rightarrow ||a|-|b||\leq |a-b| \leq |a|+|b|$
\end{center}
How $a,b\in\mathbb{R}$ and we know that $\mathbb{R}$ is an ordered set, then, by definition, one and only one that following statements is true:
\begin{enumerate}
\item $a>b$
\item $a=b$
\item $a<b$
\end{enumerate}
Suppose that $a\geq b$
\begin{enumerate}
\item thus $a\geq0\Rightarrow b\geq0$. $a\in\mathbb{R},|a|=a, \forall a\geq0$ and $|a|=-a, \forall a<0$ the same with $b$, this way we have $|a|=a$ and $|b|=b$ what tells us $||a|-|b||=|a-b|$. How $a\geq b\Rightarrow a-b>0$, then $|a-b|=a-b$, by other hand, $a\geq0,b\geq0$ again $|a|=a$ and $|b|=b\Rightarrow|a|+|b|=a+b$, thus $a-b\leq a+b\Rightarrow|a-b|\leq|a|+|b|$.
\begin{center}
$||a|-|b||=|a-b|\leq |a|+|b|$
\end{center}
\item thus $a<0 \Rightarrow b<0$, that way, we have $|a|=-a$ and $|b|=-b \Rightarrow ||a|-|b||=|-a-(-b)|=|-a+b| \Rightarrow b-a>0 \Rightarrow |b-a|=b-a \text{ and how }a-b<0 \text{ then }|a-b|=-(a-b)=-a+b=b-a$ thus $||a|-|b||=|a-b|$. Now, we already know that $|a-b|=b-a=-(a-b)$, $|a|=-a$ and $|b|=-b \Rightarrow |a|+|b|=-a-b=-(a+b) \Rightarrow -(a-b)\leq-(a+b)\Rightarrow |a-b|\leq|a|+|b|$.
\begin{center}
$||a|-|b||=|a-b|\leq |a|+|b|$
\end{center}
\end{enumerate}
\end{document}
