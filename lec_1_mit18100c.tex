\documentclass[11pt, a4paper]{amsart}
\usepackage{amsrefs}

% Language setting
% Replace `english' with e.g. `spanish' to change the document language
\usepackage[english]{babel}

% Set page size and margins
% Replace `letterpaper' with`a4paper' for UK/EU standard size
%\usepackage[a4paper,top=2cm,bottom=2cm,left=3cm,right=3cm,marginparwidth=1.75cm]{geometry}

% The following causes equations to be numbered within sections
\numberwithin{equation}{section}
% We’ll use the equation counter for all our theorem environments, so
% that everything will be numbered in the same sequence.
%plain
%Theorem, Lemma, Corollary, Proposition, Conjecture,Criterion, Assertion
%definition
%Definition, Condition, Problem, Example, Exercise,Algorithm, Question, Axiom, Property, Assumption,Hypothesis
%remark
%Remark, Note, Notation, Claim, Summary,Acknowledgment, Case, Conclusion
%       Theorem environments
\theoremstyle{plain} %% This is the default, anyway
\newtheorem{thm}{Theorem}[section]
\newtheorem{cor}{Corollary}[section]
\newtheorem{lem}{Lemma}[section]
\newtheorem{prop}{Proposition}[section]
\newtheorem{conj}{Conjecture}[section]
\newtheorem{cri}{Criterion}[section]
\newtheorem{asser}{Assertion}[section]

\theoremstyle{definition}
\newtheorem{defn}{Definition}[section]
\newtheorem{ex}{Example}[section]
\newtheorem{cond}{Condition}[section]
\newtheorem{prob}{Problem}[section]
\newtheorem{exer}{Exercise}[section]
\newtheorem{ques}{Question}[section]
\newtheorem{axi}{Axiom}[section]
\newtheorem{fact}{Fact}[section]
\newtheorem{property}{Property}[section]
\newtheorem{hyp}{Hypothesis}[section]
\newtheorem{alg}{Algorithm}[section]
\newtheorem{assump}{Assumption}[section]

\theoremstyle{remark}
\newtheorem{rem}{Remark}[section]
\newtheorem{notation}{Notation}[section]
\newtheorem{terminology}{Terminology}[section]
\newtheorem{note}{Note}[section]
\newtheorem{claim}{Claim}[section]
\newtheorem{summary}{Summary}[section]
\newtheorem{case}{Case}[section]
\newtheorem{acknow}{Acknowledgment}[section]

% Useful packages
\usepackage{amsmath}
\usepackage{amsfonts}
\usepackage{amssymb}
\usepackage{amsthm}
\usepackage{graphicx}
\usepackage[colorlinks=true, allcolors=blue]{hyperref}
\usepackage{blindtext}
\usepackage{multicol}

\title{Principles of Mathematical Analysis\\Notes of the Lecture 1\\MIT - 18.100C}
\author{Clóvis W. Bertholini Sb.}

\begin{document}
\begin{figure}[h]
%\centering
\includegraphics[width=0.25\textwidth]{mitocw.jpg}
\end{figure}

\maketitle

\begin{abstract}
Lecture 1. Sets. Ordered Sets. Examples. Ordering pairs of numbers. Largest element (maximum) and smallest element (minimum) of a subset of an ordered set.
\end{abstract}

\section{Sets}
Before we start with the mathematical analysis, we need review some topics of the set theoretic. Lets start by the definition of a general set, after that we'll go define the most important numerical sets. We'll also define what is an ordered set and their consequences as a maximum and minimum elements in its subset and ordering pairs of elements, resulted by the operations with this sets.

A set $\mathcal{S}$ is a collection of objects $x$, which each object has a common property $p$. 
We can take the set of people who has black eyes ($x$ is the people and $p$ is the property "has black eyes" or the set planets which aren't in ours galaxy ($x$ is the planets and $p$ is the property "the planet must be out of our galaxy"). Obviously we have no interest about that kind of sets.

The sets we need study here are collections of mathematical objects, as numbers, functions, sequences, series, metric spaces among others. So, lets take a more formal definition of a set,

\begin{defn}
    Given $\mathcal{S}=\lbrace x\mid x \text{ has a property } p\rbrace$, we say that $\mathcal{S}$ is a set which its elements are all objects $x$ with the property $p$.
\end{defn}

Let $\mathcal{S}$ above, if $x$ has the property $p$, we say that $x$ is an element of $\mathcal{S}$ or, more exactly, $x$ is in $\mathcal{S}$, and we formally write,
\begin{center}
    $x\in\mathcal{S}$
\end{center}

Otherwise, if $x \text{ has no the property} p$, then we say that $x$ is not in $\mathcal{S}$ write,
\begin{center}
    $x\notin\mathcal{S}$
\end{center}

\begin{ex}
Let $x$ be any geometric polygon shapes (triangles, squares, rectangles, \dots) and $S=\lbrace x \mid \text{number of vertices of }x \leq 4\rbrace$, rewrite $\mathcal{S}$ with the name of its elements.

First of all, we need find the elements of $\mathcal{S}$. Then lets do it:
\begin{itemize}
  \item A triangle has $3$ vertices, how $3\leq4$ is true, then triangle $\in \mathcal{S}$.
  \item A square has $4$ vertices, how $4\leq4$ is true, then square $\in \mathcal{S}$.
  \item A rectangle has $4$ vertices, how $4\leq4$ is true, then rectangle $\in \mathcal{S}$.
  \item A trapezium has $4$ vertices, how $4\leq4$ is true, then trapezium $\in \mathcal{S}$.
  \item A parallelogram has $4$ vertices, how $4\leq4$ is true, then parallelogram $\in \mathcal{S}$.
  \item A rhombus has $4$ vertices, how $4\leq4$ is true, then rhombus $\in \mathcal{S}$.
  \item A kite has $4$ vertices, how $4\leq4$ is true, then kite $\in \mathcal{S}$.
  \item A pentagon has $5$ vertices, how $5\leq4$ is false, then pentagon $\notin\mathcal{S}$.
  \item The other polygons has more than 5 vertices, than there isn't any one more element with number of vertices $\leq4$, than they $\notin\mathcal{S}$.
\end{itemize}

That way, we have,
\begin{center}
    $\mathcal{S}=\lbrace\textit{triangle}, \textit{square}, \textit{rectangle}, \textit{trapezium}, \textit{parallelogram}, \textit{rhombus}, \textit{kite}\rbrace$
\end{center}
\end{ex}
The example above shows us that we can describe a set in two ways, generally or specific, the specific way is good to work with sets whose a few elements, for sets with a great number of elements, generally way is more appropriated.

Sometimes, there is no one element that satisfies a property $p$. In this case, we are in front of a set $\mathcal{S}$ with no one element. It is called the \textit{empty set} and denoted by $\varnothing$, so

\begin{defn}
Let $\mathcal{S}$ a set,
\begin{center}
    $\forall x$, we have $x \notin \mathcal{S}\Rightarrow \mathcal{S}=\varnothing$.
\end{center}
\end{defn}

Lets start with numeric sets, the basic mathematical objects. There are many we already know as the natural numbers or integer numbers, also real numbers and complex numbers. We'll begin with the natural numbers.

\begin{defn}
    The natural numbers set is represented by the symbol $\mathbb{N}$, such that its elements starts at number $1$ and the next elements are always the previous element plus $1$ \textit{ad infinitum},
    \begin{center}
        $\mathbb{N}=\lbrace1,(1)+1,(2)+1,(3)+1,(4)+1,\dots\rbrace$
    \end{center}
    This way,
    \begin{center}
        $\mathbb{N}=\lbrace1,2,3,4,5,\dots\rbrace$
    \end{center}
\end{defn}
\begin{defn}
The set of the integer numbers, which is represented by the capital letter $\mathbb{Z}$, is constructed from $\mathbb{N}$, we must include at the set of the natural numbers the number $0$ and the negative numbers, to do it, just needed subtract $1$ of the first element of the natural numbers and so on, thus
\begin{center}
    $\mathbb{Z}=\lbrace\dots,(-4)-1, (-3)-1, (-2)-1, (-1)-1, (0)-1, (1)-1, \underbrace{1, 2, 3, 4, 5, \dots}_{\mathbb{N}}\rbrace$
\end{center}

Now, its easy to see $\mathbb{Z}$,

\begin{center}
    $\mathbb{Z}=\lbrace\dots, -5, -4, -3, -2, -1, 0, 1, 2, 3, 4, 5, \dots\rbrace$
\end{center}
\end{defn}

\begin{defn}
The set of rational numbers, which is represented by capital letter $\mathbb{Q}$ is composed by fractions like $p/q$, where $p$ and $q$ are integer numbers, with always $q$ different of zero,
\begin{center}
    $\mathbb{Q}=\lbrace p/q\mid p\in\mathbb{Z}, q\in \mathbb{Z}, q\neq0\rbrace$ 
\end{center}
\end{defn}

\section{Ordered Sets}

\begin{defn}
Let $\mathcal{S}$ be a set. An \textit{order} on $\mathcal{S}$ is a relation, denoted by $<$, with the following two properties:
\begin{enumerate}
    \item $x \in \mathcal{S} \Rightarrow \text{one and only one of the following statements is true:}$
    \begin{enumerate}
        \item $x<y$;
        \item $x=y$;
        \item $y<x$.
    \end{enumerate}
    \item $x<y \text{ and } y<z, x,y,z \in \mathcal{S} \Rightarrow x<z$.
\end{enumerate}
\end{defn}

The statement $x<y$ may be read as \textit{"$x$ is less than $y$"} or \textit{$x$ is smaller than $y$"} or, also, \textit{"$x$ precedes $y$"}. It is often convenient to write $y>x$ in place of $x<y$.

The notation $x\leq y$ indicates that $x<y$ or $x=y$, without specifying which of these two is to hold. In other words, $x\leq y$ is the negation of $x>y$.

\begin{defn}
An \textit{ordered set} is a set $\mathcal{S}$ in which an order is defined and is denoted by $(\mathcal{S}, <)$.
\end{defn}

\begin{ex}
$\mathbb{Q}$ is an ordered set if $\forall r,s \in \mathbb{Q}, r<s$ is defined to mean that $s-r$ is a positive rational number.
\end{ex}

\begin{defn}[Bounded Above]
Suppose $\mathcal{S}$ is an ordered set. Let $\mathcal{E}\subset\mathcal{S}$. If exists $\beta \in \mathcal{S}$ such that $x \leq \beta$ for all $x$ in $\mathcal{E}$, we say that $\mathcal{E}$ is bounded above and call $\beta$ an upper bound of $\mathcal{E}$. In mathematical words:

\begin{center}
    $\exists \beta\in\mathcal{S} \mid x\leq\beta, \forall x\in\mathcal{E},(\mathcal{S},<),\mathcal{E}\subset\mathcal{S}\Rightarrow\mathcal{E}$ is bounded above and $\beta$ is an upper bound of $\mathcal{E}$.
\end{center}
\end{defn}
More easily, if exists at least one element in $\mathcal{S}$ which is bigger than all elements of $\mathcal{E}$, this element is an upper bound of $\mathcal{E}$ and we say that $\mathcal{E}$ is bounded above.

\begin{defn}[Bounded Below]
Suppose $\mathcal{S}$ is an ordered set. Let $\mathcal{E}\subset\mathcal{S}$. If exists $\beta \in \mathcal{S}$ such that $\beta\leq x$ for all $x$ in $\mathcal{E}$, we say that $\mathcal{E}$ is bounded below and call $\beta$ a lower bound of $\mathcal{E}$. In mathematical words:

\begin{center}
    $\exists \beta\in\mathcal{S} \mid \beta\leq x, \forall x\in\mathcal{E},(\mathcal{S},<),\mathcal{E}\subset\mathcal{S}\Rightarrow\mathcal{E}$ is bounded below and $\beta$ is a lower bound of $\mathcal{E}$.
\end{center}
\end{defn}
This means, if exists at least one element in $\mathcal{S}$ which is smaller than all elements of $\mathcal{E}$, this element is a lower bound of $\mathcal{E}$ and we say that $\mathcal{E}$ is bounded below.


\begin{bibdiv}
        \begin{biblist}
            \bib{Rudin}{book}{
                author={Walter Rudin},
                title={Principles of Mathematical Analysis},
                date={1976},
                series={International Series in Pure and Applied Mathematics},
                publisher={McGraw-Hill},
                address={New York - St. Louis - San Francisco - Auckland - Bogotá - Caracas - Lisbon - London - Madrid - Mexico City - Milan - Montreal - New Deli - San Juan - Singapore - Sydney - Tokio - Toronto},
                isbn={0-07-054235-X}
            }
            \bib{Elon}{book}{
                author={Elon Lages Lima},
                title={Curso de Análise},
                date={2002},
                series={Projeto Euclides},
                volume={1},
                publisher={Instituto de Matemática Pura e Aplicada - IMPA},
                address={Rio de Janeiro},
                isbn={85-244-0116-9}
            }
            \bib{Yellowbook}{book}{
                author={Murray H. Protter},
                author={Charles B. Morrey},
                title={A first course in real analysis},
                date={1991},
                series={Undergraduate texts in mathematics},
                publisher={Springer-Verlag},
                address={New York - Berlin - Heidelberg},
                isbn={0-387-97437-7}
            }
        \end{biblist}
    \end{bibdiv}

\end{document}
